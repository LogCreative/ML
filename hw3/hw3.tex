%&"../ml"
\begin{document}
    \title{第三次作业}
    \maketitle
    \section{SVM 与神经网络}

    \subsection{数据集信息}

    对两个数据集进行测试,其规模如表 \ref{tab:dataset} 所示。其中 madelon 数据集特征维度多,训练集大小大于测试集大小;ijcnn1 数据集特征维度相对较少,但是数据集规模大,测试集大小大于训练集大小。

    \begin{table}[ht]
        \centering
        \caption{数据集信息}\label{tab:dataset}
        \begin{tabular}{crrr}
            \toprule
            数据集   & 训练集大小 & 测试集大小 & 特征维度 \\
            \midrule
            madelon & 2000 & 600 & 500 \\
            ijcnn1 & 49990 & 91701 & 22 \\
            \bottomrule
        \end{tabular}
    \end{table}

    \subsection{与 MLP 的比较}

    数据读取实现于 \filelink{src/utils.py},特征将会被首先归一化再进行训练。SVM 实现于 \filelink{src/svm.py},具体参数为
    
    \codeseg[language=python]{src/svm.py}{12}{12}
    
    MLP 实现于 \filelink{src/mlp.py},具体参数为
    
    \codeseg[language=python]{src/mlp.py}{12}{12}
    
    表 \ref{tab:mlp} 展示了默认参数下 SVM 与 MLP 的效果。从准确率来看,SVM 的准确率在 madelon 多特征维度数据集上略高于 MLP,在 ijcnn1 向本较多的数据集上低于 MLP。训练时间上 SVM 收敛需要的时间也偏长,当然从后文可以看到这个时间可以通过调节参数的方式缩短。
    
    \begin{table}[ht]
        \centering
        \caption{SVM 与 MLP}\label{tab:mlp}
        \begin{tabular}{ccccc}
            \toprule
            \multirow{2}{*}{数据集} & \multicolumn{2}{c}{准确率} & \multicolumn{2}{c}{训练时间 (s)} \\
            \cmidrule(r){2-3}\cmidrule(l){4-5}
            & SVM & MLP & SVM & MLP \\
            \midrule
            madelon & \textbf{0.585} & 0.583 & 71 & 19 \\
            ijcnn1 & 0.919 & \textbf{0.961} & 130 & 91 \\
            \bottomrule
        \end{tabular}
    \end{table}

    \subsection{不同的核函数}

    表 \ref{tab:kernel} 展示了使用不同核函数的结果。其中在多维度的 madelon 上 linear 核的表现最好,但是训练时间较长,使用其它核会略微降低一点准确率,但是时间可以减少一个数量级。在少一些维度的 ijcnn1 上,rbf 的准确率最高,可以超过表 \ref{tab:mlp} 的 MLP,此时的 poly 核是更好的性价比选择。

    \begin{table}[ht]
        \centering
        \caption{SVM 不同核函数,$C=1$}\label{tab:kernel}
        \begin{tabular}{ccccccccc}
            \toprule
            \multirow{2}{*}{数据集} & \multicolumn{4}{c}{准确率} & \multicolumn{4}{c}{训练时间 (s)} \\
            \cmidrule(r){2-5}\cmidrule(l){6-9}
            & linear & poly & rbf & sigmoid & linear & poly & rbf & sigmoid \\
            \midrule
            madelon & \textbf{0.585} & 0.578 & 0.582 & 0.583 & 76 & 5  & 7 & 5 \\
            ijcnn1 & 0.919 & 0.948 & \textbf{0.968} & 0.867 & 150 & 70 & 141 & 127 \\
            \bottomrule
        \end{tabular}
    \end{table}

    \subsection{不同的 $C$}

    表 \ref{tab:c} 展示了不同的 $C$ 对 linear SVM 的影响,$C$ 将控制对软间隔的容忍度。可见其对准确率不会有特别大的影响,但是在训练时间上会有差异。同等准确率的情况下,对 madelon 而言,$C=0.1$ 最好;对 ijcnn1 而言,$C=0.1$ 最好。

    \begin{table}[ht]
        \centering
        \caption{不同的 $C$,linear}\label{tab:c}
        \begin{tabular}{ccccccccc}
            \toprule
            \multirow{2}{*}{数据集} & \multicolumn{4}{c}{准确率} & \multicolumn{4}{c}{训练时间 (s)} \\
            \cmidrule(r){2-5}\cmidrule(l){6-9}
            & 0.01 & 0.1 & 0.5 & 1 & 0.01 & 0.1 & 0.5 & 1 \\
            \midrule
            madelon & 0.568 & \textbf{0.585} & 0.57 & \textbf{0.585} & 5 & 9  & 29 & 71 \\
            ijcnn1 & 0.918 & \textbf{0.919} & \textbf{0.919} & \textbf{0.919} & 87 & 104 & 121 & 188 \\
            \bottomrule
        \end{tabular}
    \end{table}

\end{document}